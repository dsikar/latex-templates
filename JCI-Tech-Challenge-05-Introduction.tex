\section{Introduction}

The impact of AI on industry has been compared to the advent of commodity electricity 100 years ago \cite{ShanaLynch}. At the heart of such claim is the fact that our lives are becoming increasingly complex and the resulting management overhead closer and closer to intractable.  
In this period of flux, threats and opportunities co-exist and it is imperative that JCI evolve and adapt to this new era of smart homes and smart cities, bringing both together coherently. Models have been proposed, deploying technologies such as 5G, the Internet of Things, cloud of things and distributed artificial intelligence. 

This new infrastructure offers a number of benefits and service options, such as interconnected Internet of Things, intelligent intelligent homes, and a platform for new combined intelligent home and intelligent urban services based on large data \cite{7019822}. To strengthen its position of leadership, JCI must pro-actively incorporate AI research into current projects, avoiding the risk of losing ground to competitors.  

\subsection{Target Audience}

This text is aimed at the current JCI workforce currently in engineering and management roles, who are interested in incorporating AI into products and services.

\subsection{Scope}

This text covers AI superficially and although some references are provided, should not be taken verbatin as the field is in flux and equally authoritative views are occasionally contradictory. The authors wish to initiate an internal debate and propose one initial course of action to bring AI into ongoing and future R\&D projects.




