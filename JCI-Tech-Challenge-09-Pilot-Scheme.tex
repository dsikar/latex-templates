\section{Pilot Scheme}

The authors propose a pilot scheme to gather data, develop and deploy AI models and ultimately complete a patent application of the process, given "Deep learning (...) now constitute 63.2 percent" of patents in AI it is hardly new territory \cite{Wipo2019} as far as patent applications apply.

Additionally, a second front is suggested examining possibilities with signal processing and filtering, given the infrastructure is similar for both cases.

A time and headcount allocation is proposed over a 2-year cycle:

\begin{table}[ht]
\centering
\begin{tabular}{|l|l|l|l|}
\hline
\multicolumn{4}{|l|}{Roles, headcount and hours} \\ \hline
Role  &  Head count & Hours (weekly) & Hours (Annualy) \\ \hline
Software Engineering Intern & x2 & 2x5h ~ 10h  & 10hx46w ~ 460h\\ \hline
Project Manager & x1 & 1x5h & 5x46w ~ 230h \\ \hline
\multicolumn{3}{|1|}{Partial annual time commitment total} & 690h \\ \hline
\multicolumn{3}{|1|}{Total 2-year time commitment} & 1380h \\ \hline


\end{tabular}
\caption{JCI AI Pilot Scheme time commitment table}
\label{table:kysymys}
\end{table}

The deliverables being:

\begin{itemize}
    \item A trained model for assisted component buying in the form of a phone app
    \item At least one patent to cover the process
\end{itemize}

Notice the patent is a form of protection against competitors using the same pipeline to produce similar models. It is not our primary purpose to turn this particular deliverable into a commercial product. 

Checkpoints are proposed after the one and two year mark respectively, to evaluate the progress and adjust direction if needed.



